\section{Framing}

\begin{frame}{Why this deck}
  \begin{itemize}
    \item How do we explain incentives without “expect profit” language?
    \item Show how operator profitability emerges (math + drivers), including break-even.
    \item Tie tokenomics to security (stake + disputes should cover what we secure).
    \item Position TRB as utility (gas, staking, disputes).
  \end{itemize}
\end{frame}

\begin{frame}{Messaging guardrails (internal)}
  \begin{itemize}
    \item Use: “earnings for providing service”, “variable”, “costs + risk”, “competitive market”.
    \item Avoid: “guaranteed yield”, “passive income”, “buy TRB to profit”.
    \item Keep profitability numbers framed as operator P\&L, not investor return.
  \end{itemize}
\end{frame}

\begin{frame}{Mental model}
  \titlebox{Layer is a service marketplace}{
    Users/tippers pay for data \arrow reporters provide data \arrow validators provide base security.
  }
  \vspace{0.6em}
  \begin{itemize}
    \item Two reward sources: time-based rewards + tips.
    \item Security enforcement: disputes with fees, voting, and slashing/jailing.
  \end{itemize}
\end{frame}

\begin{frame}{Units \& terms}
  \begin{itemize}
    \item Base denom: \texttt{loya} (1 TRB = 1,000,000 loya).
    \item Reporter “power” is effectively stake expressed in whole-TRB units.
    \item A report has costs (tx fees) and potential revenue (tips + time-based rewards).
  \end{itemize}
\end{frame}

