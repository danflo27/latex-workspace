\section{Mechanisms}

\begin{frame}{Where rewards come from (overview)}
  \begin{itemize}
    \item Time-based rewards: inflation + optional bootstrap “extra rewards” pool.
    \item Tips: user-funded; 2\% burned; net goes to escrow-backed entitlements.
    \item Disputes: fees, burns, voter rewards, slashing/jailing (security lever).
  \end{itemize}
\end{frame}

\begin{frame}{Time-based rewards (high level)}
  \begin{itemize}
    \item Inflationary rewards are distributed continuously (block-time proportional).
    \item Split: \textbf{75\%} to reporter incentive stream; \textbf{25\%} to validator distribution stream.
    \item Framing: incentives for operating / securing / servicing, not guaranteed “yield”.
  \end{itemize}
\end{frame}

\begin{frame}{Bootstrap extra rewards}
  \begin{itemize}
    \item Extra rewards are pre-funded and distributed at a configured rate.
    \item Use this slide to explain early-phase bootstrap vs steady-state.
    \item \textbf{PLACEHOLDER}: current pool size, runway estimate (pull from profitability checker).
  \end{itemize}
\end{frame}

\begin{frame}{Tips: burn + escrow (why it exists)}
  \begin{itemize}
    \item 2\% tip burn \arrow cost of influence / discourages spam.
    \item Net tips are escrow-backed until withdrawn.
    \item Tips create a demand-driven reward component (not only emissions).
  \end{itemize}
\end{frame}

\begin{frame}{Disputes (security lever)}
  \begin{itemize}
    \item Fees scale by dispute level; outcomes can trigger slashing/jailing.
    \item A portion is burned; a portion becomes voter reward pool.
    \item \textbf{PLACEHOLDER}: dispute activity stats (count, size, outcomes).
  \end{itemize}
\end{frame}

\begin{frame}{Security coverage lens (setup)}
  \begin{itemize}
    \item Security comes from bonded stake at risk + credible dispute penalties.
    \item Define: coverage = stake-at-risk / value-secured.
    \item \textbf{PLACEHOLDER}: define value-secured per integration (TVL, notional, etc.).
  \end{itemize}
\end{frame}

